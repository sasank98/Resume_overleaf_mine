%-------------------------
% Resume in Latex
% Author : Jake Gutierrez
% Based off of: https://github.com/sb2nov/resume
% License : MIT
%------------------------

\documentclass[letterpaper,10.5pt]{article}

\usepackage{latexsym}
\usepackage[empty]{fullpage}
\usepackage{titlesec}
\usepackage{marvosym}
\usepackage[usenames,dvipsnames]{color}
\usepackage{verbatim}
\usepackage{enumitem}
\usepackage[hidelinks]{hyperref}
\usepackage{fancyhdr}
\usepackage[english]{babel}
\usepackage{tabularx}
\usepackage{fontawesome5}
\usepackage{multicol}
\setlength{\multicolsep}{-3.0pt}
\setlength{\columnsep}{-1pt}
\input{glyphtounicode}


%----------FONT OPTIONS----------
% sans-serif
% \usepackage[sfdefault]{FiraSans}
% \usepackage[sfdefault]{roboto}
% \usepackage[sfdefault]{noto-sans}
% \usepackage[default]{sourcesanspro}

% serif
% \usepackage{CormorantGaramond}
% \usepackage{charter}


\pagestyle{fancy}
\fancyhf{} % clear all header and footer fields
\fancyfoot{}
\renewcommand{\headrulewidth}{0pt}
\renewcommand{\footrulewidth}{0pt}

% Adjust margins
\addtolength{\oddsidemargin}{-0.6in}
\addtolength{\evensidemargin}{-0.5in}
\addtolength{\textwidth}{1.19in}
\addtolength{\topmargin}{-.7in}
\addtolength{\textheight}{1.1in} %1.22 for parallel

\urlstyle{same}

\raggedbottom
\raggedright
\setlength{\tabcolsep}{0in}

% Sections formatting
\titleformat{\section}{
  \vspace{-4pt}\scshape\raggedright\large\bfseries
}{}{0em}{}[\color{black}\titlerule \vspace{-5pt}]

% Ensure that generate pdf is machine readable/ATS parsable
\pdfgentounicode=1

%-------------------------
% Custom commands
\newcommand{\resumeItem}[1]{
  \item\small{
    {#1 \vspace{-2pt}}
  }
}

\newcommand{\classesList}[4]{
    \item\small{
        {#1 #2 #3 #4 \vspace{-2pt}}
  }
}

\newcommand{\resumeSubheading}[4]{
  \vspace{-2pt}\item
    \begin{tabular*}{1.0\textwidth}[t]{l@{\extracolsep{\fill}}r}
      \textbf{#1} & \textbf{\small #2} \\
      \textit{\small#3} & \textit{\small #4} \\
    \end{tabular*}\vspace{-7pt}
}

\newcommand{\resumeSubSubheading}[2]{
    \item
    \begin{tabular*}{0.97\textwidth}{l@{\extracolsep{\fill}}r}
      \textit{\small#1} & \textit{\small #2} \\
    \end{tabular*}\vspace{-7pt}
}

\newcommand{\resumeProjectHeading}[2]{
    \item
    \begin{tabular*}{1.001\textwidth}{l@{\extracolsep{\fill}}r}
      \small#1 & \textbf{\small #2}\\
    \end{tabular*}\vspace{-7pt}
}

\newcommand{\resumeSubItem}[1]{\resumeItem{#1}\vspace{-4pt}}

\renewcommand\labelitemi{$\vcenter{\hbox{\tiny$\bullet$}}$}
\renewcommand\labelitemii{$\vcenter{\hbox{\tiny$\bullet$}}$}

\newcommand{\resumeSubHeadingListStart}{\begin{itemize}[leftmargin=0.0in, label={}]}
\newcommand{\resumeSubHeadingListEnd}{\end{itemize}}
\newcommand{\resumeItemListStart}{\begin{itemize}}
\newcommand{\resumeItemListEnd}{\end{itemize}\vspace{-5pt}}

%-------------------------------------------
%%%%%%  RESUME STARTS HERE  %%%%%%%%%%%%%%%%%%%%%%%%%%%%


\begin{document}

%----------HEADING----------
% \begin{tabular*}{\textwidth}{l@{\extracolsep{\fill}}r}
%   \textbf{\href{http://sourabhbajaj.com/}{\Large Sourabh Bajaj}} & Email : \href{mailto:sourabh@sourabhbajaj.com}{sourabh@sourabhbajaj.com}\\
%   \href{http://sourabhbajaj.com/}{http://www.sourabhbajaj.com} & Mobile : +1-123-456-7890 \\
% \end{tabular*}

\begin{center}
    {\Huge \scshape Sasank Potluri} \\ \vspace{1pt}
%    123 Street Name, Town, State 12345 \\ \vspace{1pt}
    \small \raisebox{-0.1\height}\faPhone\ +1 (617)-792-6969 ~ 
    \href{mailto:sasank4496@gmail.com}{\raisebox{-0.2\height}\faEnvelope\ \underline{sasank4496@gmail.com}} ~ 
    \href{https://www.linkedin.com/in/sasank-potluri-991496180/}{\raisebox{-0.2\height}\faLinkedin\ \underline{sasank-potluri-991496180/}}  ~
    \href{https://github.com/sasank98}{\raisebox{-0.2\height}\faGithub\ \underline{sasank98}}
    \vspace{-8pt}
\end{center}


%-----------EDUCATION-----------
\section{EDUCATION}
  \resumeSubHeadingListStart
    \resumeSubheading
      {Master of Science in Robotics and automation}{Jan 2022 -- May 2024} 
      {Northeastern Khoury college of Computer Science}{3.81/4 GPA}
      \vspace{3pt}
      \newline{Relevant Courses: Computer Vision, Factor Graphs, Sensor Fusion, Deep Learning, Reinforcement Learning, Control Systems}
    \vspace{-4pt}
%  \resumeSubHeadingListEnd
%  \resumeSubHeadingListStart
    \resumeSubheading
      {Bachelor of Technology in Mechanical Engineering}{Jul 2016 -- Jun 2020}
      {Manipal Institute of Technology}{8.29/10 GPA}
      \vspace{3pt}
      \newline{Relevant Courses: Manufacturing Technology, Computer Aided Drawing, Machine Design, Nonlinear Optimization}
  \resumeSubHeadingListEnd
\vspace{-19pt}
%------RELEVANT COURSEWORK-------
% \section{Relevant Coursework}
%     %\resumeSubHeadingListStart
%         \begin{multicols}{4}
%             \begin{itemize}[itemsep=-5pt, parsep=3pt]
%                 \item\small Data Structures
%                 \item Software Methodology
%                 \item Algorithms Analysis
%                 \item Database Management
%                 \item Artificial Intelligence
%                 \item Internet Technology
%                 \item Systems Programming
%                 \item Computer Architecture
%             \end{itemize}
%         \end{multicols}
%         \vspace*{2.0\multicolsep}
    %\resumeSubHeadingListEnd

%-----------EXPERIENCE-----------
\section{EXPERIENCE}
  \resumeSubHeadingListStart
    \resumeSubheading
      {Autonomy Systems Intern $|$ \textmd{\emph{C++, Python, PyTorch, Azure, CANalyzer}}}{Jan 2023 -- Aug 2023}
      {Danfoss Autonomy}{Minneapolis, Minnesota}
      \resumeItemListStart
        \resumeItem{Adapted LIO-SAM for a 6-axis IMU using additional GPS and conducted system testing to identify off-road failure modes}
        \resumeItem{Leveraged ROS to test third party SLAM algorithms and LIO-SAM on robot hardware for company specific use cases}
        \resumeItem{Conducted System Testing along with FMEAs on third party SLAM algorithms and LIO-SAM for company specific use cases}
        
        \resumeItem{Developed C++ code to provide ethernet communication between a SLAM controller and Danfoss controller}
        \resumeItem{Implemented an additional decoder on unimatch network to parallelly compute Optic Flow, Stereo Disparity, and 3D detection}
        \resumeItem{Achieved a 9.87 AP3D on Object detection task without fine-tuning the encoder of the unimatch network}
        \resumeItem{Performed 3D-object detection using YOLO-v8 and DBSCAN on the colored pointcloud generated from LiDAR camera fusion}
        % \resumeItem{Used synthetic data from Nvidia-IsaacSim to train YOLO-v8 to detect forklifts and achieved mAP of 0.53 on real data}
        \resumeItem{Leveraged synthetic data from Nvidia-IsaacSim to train YOLO-v8, achieved 0.45 mAP for real-world forklift detection}
%        \resumeItem{Implemented a novel architecture on Stereo Cameras to compute Flow, Stereo Disparity and 3D-object detection simultaneously}
        \resumeItem{Fused a 16-ch 2D LiDAR with camera data using a depth completion network, resulting in an MAE of 0.178 meters}
        % \resumeItem{Explored different methods to implement a global path planner on the generated map data from LiDAR SLAM algorithms}

      \resumeItemListEnd

    \resumeSubheading
      {Research Assistant at Hydrodynamics Lab $|$ \textmd{\emph{Ansys, SolidWorks, Fusion360, Matlab}}}{Jan 2020 -- Jun 2020}
      {Manipal Institute of Technology}{Manipal, Karnataka}
      \resumeItemListStart
        % \resumeItem{generated CNC code to add another pressure sensor in the testing setup}
        \resumeItem{Added additional Pressure sensor to existing test-rig and acquired dynamic pressure and position readings using Matlab}
        %\resumeItem{3D-printed and added composite reinforcement to Intake Manifold achieving over 50\% weight reduction compared to previous versions}
        \resumeItem{Used the Data acquisition system to acquire the dynamics and stability of a water-lubricated hydrodynamic bearing}
        \resumeItem{Created a 3D dynamic CFD model in Ansys and used the data to tune simulation parameters}
        %\resumeItem{Made changes to the existing model to improve the dynamic stability of the bearing but achieved only similar performance}
        \resumeItem{Revamped the model to boost dynamic bearing stability, but achieved performance similar to the original design}
    \resumeItemListEnd
    
  \resumeSubHeadingListEnd
\vspace{-16pt}

%-----------PROJECTS-----------
\section{PROJECTS}
    \vspace{-5pt}
    \resumeSubHeadingListStart
      \resumeProjectHeading
          {\textbf{Bundle adjustment on Buddha images} $|$ \emph{Python, GTSAM, OpenCV}}{Nov 2023}
          \resumeItemListStart
            \resumeItem{Implemented SFM pipeline for sparse 3D reconstruction from images, used SIFT to extract, match and triangulate 3D keypoints}
%            \resumeItem{Applied bundle adjustment on traingulated keypoints using GTSAM to get optimized keypoint and pose estimates}
            \resumeItem{Solved Bundle Adjustment using GTSAM to get accurate and optimized keypoints with camera poses}
          \resumeItemListEnd
          \vspace{-15pt}
      \resumeProjectHeading
          {\textbf{3D object tracking using Multi-view Images } $|$ \emph{Python, PyTorch, NumPY}}{Nov 2023}
          \resumeItemListStart
            \resumeItem{Achieved an Object tracking accuracy of 15.1\% by implementing an Extended Kalman Filter on 3D object-detections}
            %\resumeItem{Implemented a tracking decoder on PETR-v1 model and achieved an accuracy of 20.8\%, and conducted comparative analysis}
            % \resumeItem{Implemented a tracking decoder on PETR-v1 model, achieving an accuracy of 20.8\%, and subsequently conducted a comparative analysis of both approaches}
            \resumeItem{Implemented PETR-v1 tracking decoder, achieving 20.8\% accuracy, and conducted comparative analysis of both approaches}
          \resumeItemListEnd 
          \vspace{-15pt}
          \resumeProjectHeading
          {\textbf{Feature detection and Image mosaic } $|$ \emph{Linux, Python, GTSAM}}{Sep 2023}
          \resumeItemListStart
            \resumeItem{Used the Caltech camera calibration toolbox to compute extrinsic and intrinsic parameters and undistort the images}
            %\resumeItem{Used Superglue along with other classical feature detectors to compute matches among images on an underwater archeological site}
            \resumeItem{Applied Superglue and other classical feature detectors to compute image matches in an underwater archaeological site}
            \resumeItem{Created a mosaic using the matches and optimized the pose graph to obtain better mosaic of the site}
          \resumeItemListEnd 
          \vspace{-15pt}
          \resumeProjectHeading
          {\textbf{Reinforcement Learning on Robotic Arm } $|$ \emph{Python, PyTorch}}{Nov 2022}
          \resumeItemListStart
            %\resumeItem{Iterated through different reinforcement learning algorithms and tuned its hyper-parameters to train a Robotic arm perform pick and place, and reach a point operation}
            \resumeItem{Iterated through various continuous control algorithms to train a robotic arm for pick-and-place and reach a point operations}
            \resumeItem{Re-engineered the reward function to penalize the number of moves improving speed and stability of Robotic Arm}
          \resumeItemListEnd 
          \vspace{-15pt}
          \resumeProjectHeading
          {\textbf{Reinforcement Learning on Robotic Arm } $|$ \emph{Python, PyTorch}}{Nov 2022}
          \resumeItemListStart
            %\resumeItem{Iterated through different reinforcement learning algorithms and tuned its hyper-parameters to train a Robotic arm perform pick and place, and reach a point operation}
            \resumeItem{Iterated through various continuous control algorithms to train a robotic arm for pick-and-place and reach a point operations}
            \resumeItem{Re-engineered the reward function to penalize the number of moves improving speed and stability of Robotic Arm}
          \resumeItemListEnd 
          \vspace{-15pt}
          \resumeProjectHeading
          {\textbf{Reinforcement Learning on Robotic Arm } $|$ \emph{Python, PyTorch}}{Nov 2022}
          \resumeItemListStart
            %\resumeItem{Iterated through different reinforcement learning algorithms and tuned its hyper-parameters to train a Robotic arm perform pick and place, and reach a point operation}
            \resumeItem{Iterated through various continuous control algorithms to train a robotic arm for pick-and-place and reach a point operations}
            \resumeItem{Re-engineered the reward function to penalize the number of moves improving speed and stability of Robotic Arm}
          \resumeItemListEnd 
          \vspace{-15pt}
          \resumeProjectHeading
          {\textbf{Control system design of Segway } $|$ \emph{Matlab, Simulink}}{May 2022}
          \resumeItemListStart
            %\resumeItem{Iterated through different reinforcement learning algorithms and tuned its hyper-parameters to train a Robotic arm perform pick and place, and reach a point operation}
            \resumeItem{Used stability analysis, and root locus design to pick optimum proportional, integral, and derivative values to control a segway}
            \resumeItem{Further solved the problem by using state space methods and improved segway performance by adding additional compensator}
            % \resumeItem{}
            % \resumeItem{}
          \resumeItemListEnd 
          \vspace{-15pt}
          \resumeProjectHeading
          {\textbf{Performance comparision among SLAM algorithms} $|$ \emph{ROS, C++, ORB-SLAM3, LeGO-LOAM, Matlab}}{Apr 2022}
          \resumeItemListStart
            %\resumeItem{Collected data from cameras, LiDAR, IMU and GPS Sensors of Northeastern’s self-driving car (NUANCE car) by writing a ROS publisher node in C++}
            \resumeItem{Collected camera, LiDAR, IMU and GPS data of test vehicle driving in urban environment by writing ROS publisher node in c++}
            % \resumeItem{Used the collected data to test LeGO-LOAM and ORB-SLAM3 and compared the performance and failure cases for both algorithms}
            \resumeItem{Utilized the collected data to test LeGO-LOAM and ORB-SLAM3 and compared their results and failure cases}
            \resumeItemListEnd 
          \vspace{-15pt}
      \resumeProjectHeading
          {\textbf{Structural design for FSAE racecar(Formula Manipal)} $|$ \emph{Ansys, CATIA, SolidWorks, Fusion360}}{Oct 2017 -- Feb 2019}
          \resumeItemListStart
            % \resumeItem{Implemented SFM pipeline for sparse 3D reconstruction from images, used SIFT to extract, match and triangulate 3D keypoints}
            \resumeItem{Lead the composite subsystem of the team to efficiently design and manufacture Composite automotive parts}
            \resumeItem{3D-printed, added composite reinforcement to Intake Manifold, achieving 50\%+ weight reduction compared to prior versions}
            % \resumeItem{Generated the }
            \resumeItem{Designed and manufactured Carbon-fiber seat, and aero-package and won second place for design in Formula Bharat 2019}
          \resumeItemListEnd
          %\vspace{-15pt}
          % \resumeProjectHeading
          % {\textbf{Transaction Management GUI} $|$ \emph{Java, Eclipse, JavaFX}}{October 2020}
          % \resumeItemListStart
          %   \resumeItem{Designed a sample banking transaction system using Java to simulate the common functions of using a bank account.}
          %   \resumeItem{Used JavaFX to create a GUI that supports actions such as creating an account, deposit, withdraw, list all acounts, etc.}
          %   \resumeItem{Implemented object-oriented programming practices such as inheritance to create different account types and databases.}
          % \resumeItemListEnd 
    \resumeSubHeadingListEnd
\vspace{-16pt}


%
%-----------PROGRAMMING SKILLS-----------
\section{TECHNICAL SKILLS}
 \begin{itemize}[leftmargin=0.15in, label={}]
    \small{\item{
     \textbf{Languages}{: Python, Java, C, HTML/CSS, JavaScript, SQL} \\
     \textbf{Developer Tools}{: VS Code, Eclipse, Google Cloud Platform, Android Studio} \\
     \textbf{Technologies/Frameworks}{: Linux, Jenkins, GitHub, JUnit, WordPress} \\
    }}
 \end{itemize}
 \vspace{-20pt}


%-----------INVOLVEMENT---------------
\section{ACHIEVEMENTS / EXTRACURRICULAR}
    %\resumeSubHeadingListStart
    %    \resumeSubheading{Fraternity}{Spring 2020 -- Present}{President}{University Name}
    \resumeItemListStart
        \resumeItem{Appeared for GATE 2021 and secured an All-India Rank of 523 out of 120594 (99.566 percentile)}
        \vspace{-5pt}
        \resumeItem{Secured 3rd place in FSAE Bharat 2019 and 2nd place in the design event of the competition}
        \vspace{-5pt}
        \resumeItem{Engineered carbon fiber components for FSAE race car, such as intake manifold, sidepods, and seat, achieving a minimum 13\% weight reduction compared to previous versions}
        \vspace{-5pt}
        \resumeItem{Engineered carbon fiber components for FSAE race car achieving a minimum weight reduction of 13\% on all parts }
        \vspace{-5pt}
    \resumeItemListEnd
        
    %\resumeSubHeadingListEnd

%  \begin{itemize}[leftmargin=0.15in, label={}]
%     \small{\item{
%         \resumeItemListStart
%             \resumeItem{Languages: Python, Java, C, HTML/CSS, JavaScript, SQL}
%             \resumeItem{Developer Tools: VS Code, Eclipse, Google Cloud Platform, Android Studio} 
%             \resumeItem{Technologies/Frameworks: Linux, Jenkins, GitHub, JUnit, WordPress}
%         \resumeItemListEnd
%     }}
%  \end{itemize}
%  \vspace{-20pt}

\end{document}
